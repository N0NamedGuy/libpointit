\documentclass[a4paper,oneside]{article}

\usepackage{fancyhdr}
\usepackage[pdftex,colorlinks]{hyperref}
\usepackage[portuguese]{babel}
\usepackage[utf8]{inputenc}
\usepackage[T1]{fontenc}

\selectlanguage{portuguese}

\hypersetup{
  pdfauthor={David Serrano},
  pdfkeywords={pointit, libpointit, color recognition, 
    ejc, ajc, ejc2010},
  pdftitle={libpointit},
  pdfsubject={Pointit Library},
    colorlinks=true,
  linkcolor=black,
  urlcolor=blue,
  citecolor=black,
  filecolor=black
}

\title{PointIt Library - libpointit}
       \author{David Miguel de Araújo Serrano}
%\thispagestyle{empty}
%\addtocounter{page}{-1}

\begin{document}
\maketitle

Este software é uma biblioteca que permite aos programas que a
usem fazer reconhecimento de cor de objectos, através de uma
\emph{webcam} ou de outro dispositivo equivalente.

Após a detecção das coordenadas espaciais de um objecto de determinada cor
(por omissão, verde), o programa que use {{\bf \emph{libpointit}} poderá usar
a informação das coordenadas e do tamanho do objecto em questão conforme
entender. Uma prática comum por parte desses programas clientes será a
reposição de, por exemplo, um cursor do rato.

Todas as aplicações que venham a usar a biblioteca {\bf \emph{Point It}}
poderão tomar partido de novos paradigmas no desenho de GUIs (do inglês,
\emph{Graphical User Interfaces}), de novas formas de interacção com
videojogos e uma melhor acessibilidade para pesosas com deficiências.
Estas são apenas algumas das vantagens que as aplicações que usem a
{\bf \emph{libpointit}} passam a ter, para além de outras que não foram aqui
mencionadas.

A melhor característica da {\bf \emph{libpointit}} é a possibilidade do uso
de objectos do nosso dia-a-dia. Basta usar qualquer objecto, e movê-lo em frente
à \emph{webcam} (desde que este seja da cor para a qual a {\bf \emph{libpointit}}
está configurada para reconhecer).

Esta biblioteca é \emph{open-source} (de código aberto), e completamente
livre (pode-se alterar e redistribuir, de acordo com a licença \emph{GPL})
este \emph{software}, e está disponível para download em
http://github.com/N0NamedGuy/libpointit
\end{document}
