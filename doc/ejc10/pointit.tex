\documentclass[a4paper,oneside]{article}

\usepackage{fancyhdr}
\usepackage[pdftex,colorlinks]{hyperref}
\usepackage[portuguese]{babel}
\usepackage[utf8]{inputenc}
\usepackage[T1]{fontenc}
\usepackage{hyperref}

\selectlanguage{portuguese}

\hypersetup{
  pdfauthor={David Serrano},
  pdfkeywords={pointit, libpointit, color recognition, 
    ejc, ajc, ejc2010},
  pdftitle={libpointit},
  pdfsubject={Pointit Library},
    colorlinks=true,
  linkcolor=black,
  urlcolor=blue,
  citecolor=black,
  filecolor=black
}

\title{PointIt Library - libpointit}
       \author{David Miguel de Araújo Serrano}
%\thispagestyle{empty}
%\addtocounter{page}{-1}

\begin{document}
\maketitle

Pretendo levar ao Encontro Juvenil de Ciência de 2010 um \emph{software}
inovador, em desenvolvimento activo, de nome {\bf \emph{Point It}}, também
apelidado de {\bf \emph{libpointit}}.

Sumariamente, este software é uma biblioteca que permite aos programas que a
usem fazer reconhecimento de cor de objectos, através de uma
\emph{webcam} ou de outro dispositivo equivalente.

Após a detecção das coordenadas espaciais de um objecto de determinada cor
(por omissão, verde), o programa que use a {{\bf \emph{libpointit}} poderá usar
a informação das coordenadas e do tamanho do objecto em questão conforme
entender. Uma prática comum por parte desses programas clientes será a
reposição de, por exemplo, um cursor do rato.

Todas as aplicações que venham a usar a {\bf \emph{libpointit}}
poderão tomar partido de novos paradigmas no desenho de GUIs (do inglês,
\emph{Graphical User Interfaces}), de novas formas de interacção com
videojogos e uma melhor acessibilidade para pessoas com deficiências.
Estas são apenas algumas das vantagens que as aplicações que usem a
{\bf \emph{libpointit}} passam a ter, para além de outras que não foram aqui
mencionadas.

A melhor característica da {\bf \emph{libpointit}} é a possibilidade do uso
de objectos do nosso dia-a-dia. Basta usar qualquer objecto, e movê-lo em frente
à \emph{webcam} (desde que este seja da cor para a qual a {\bf \emph{libpointit}}
está configurada para reconhecer). Nos testes executados por mim, amigos e familiares
a este software, foram usados um tubo de cola cuja ponta está revestida com um pano
verde fluoroscente e uma laterna coberta na ponta pelo mesmo pano.

Por esta biblioteca ser \emph{open-source} (de código aberto), e completamente
livre (pode-se alterar e redistribuir, de acordo com a licença \emph{GPL})
este \emph{software} torna-se uma solução barata e eficaz para a resolução de
problemas de acessibilidade e implementação de controlos baseados em movimento.

Com as melhorias que foram efectuadas desde a última versão (quando nem sequer era
uma biblioteca e se chamava \emph{nGp - next Generation pointer}) pode-se agora obter
\emph{frame-rates} extraordinárias. Na ordem das 30 \emph{frames} por segundo. Comparadas
com as anteriores 15 \emph{frames} por segundo, isto é uma melhoria na ordem dos 100\%!
Para esta melhoria da performance, fiz uma grande optimização do código.

Para a demonstração desta biblioteca, levarei também algumas demos, como por exemplo:
o jogo do Pong, o jogo do Tuka, programa de desenho, entre outras surpresas...
Algumas das demonstrações são exibidas em \url{http://www.youtube.com/view_play_list?p=AB993577A3C6A363}

Esta biblioteca foi desenvolvida para sistemas \emph{Unix-like} (\emph{Linux} e 
\emph{MacOS} por exemplo), com a biblioteca \emph{OpenCV} instalada, pois é necessária para a
aquisição de imagem.

O código da {\bf \emph{libpointit}}, assim como este documento em
formato \LaTeX \,  está disponível para download em \url{http://github.com/N0NamedGuy/libpointit}
\end{document}.
